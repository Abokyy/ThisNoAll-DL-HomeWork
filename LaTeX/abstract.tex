%\pagenumbering{roman}
\chapter*{Abstract}
\addcontentsline{toc}{chapter}{Abstract}

%This document is a \LaTeX -based template for the BSc/MSc thesis of students at the Electrical Engineering and Informatics Faculty of Budapest University of Technology and Economics. The usage of this template is optional. It has been tested with the TeXLive TEX implementation, and it requires the PDF-LaTeX compiler.



%Célom egy olyan megoldás találása, ami univerzálisan minden dokumentum típust lekezel külső beavatkozás nélkül. Ez egy olyan dobozos késztermék, amely a megkapott dokumentumokból fejezeteket állít elő teljesen automatizálva. Tetszőleges mélységig a szövegstruktúra felismerése nem müködik önállóan, viszont kiegészitve pár modul segítségével minden információ előállítható amit egy dokumentum magában hordoz. Ahhoz, hogy a dokumentumokat egységesen letudjuk kezelni, egységes rögzített formátumba kell őket leképezni.

This thesis is aimed at giving a universal solution for text structure recognition in documents. The solution attempts to address the continuously growing volume of documentation base that from a user perspective presents a challenge. Undoubtedly, processing documentation is a time consuming and demanding task. It is not uncommon to see that users choose alternative sources to address specific tasks rather than processing voluminous documents. However, this raises several concerns. For example, the reliability of alternative sources is usually poor. Consequently, they can cause more harm than good.

In an attempt to address this problem, this work gives a solution for systematic decomposition of large datasets into elements so that a search engine can process them. The outcome is the knowledge that is likely the most relevant for the user from the data mining from text structures perspective. A critical requirement for such a solution is universality, meaning that it can be applied regardless of the source format and its language. 

In this thesis, I examine several approaches that can yield the desired outcome. Eventually, I have selected the one that provides the best results and implemented in a prototype. The solution gives a search interface that can be accessed via a regular browser. Evaluations showed that my solution outperforms other tools included in the comparison. Furthermore, I also show what useful statistical features can be extracted from the data and their applications. I conclude this thesis with options for further imporvements and future research implications.